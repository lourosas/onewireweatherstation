\documentclass[letterpaper]{article}
\usepackage{latexsym}
\usepackage{amsbsy}
\usepackage{graphicx}
\begin{document}
\noindent
\textbf{Use Case S5:  Detailed Description}\\\\
Use Case Name:  Monitor Heat Index Data\\
Scenario:  Record  Daily Heat Index  Extremes\\
Brief Description:  With the System running, the System records the
heat index extremum data--the minimum and maximum heat index--from the
calculated heat index data.\\
Actors:  System\\
Related Use Cases:\\
\textbf{Use Case S15:  }\emph{The System Shall Save Heat Index
Extremum}\\
\textbf{Stakeholders:  }Local and National Weather Bureaus and
individuals monitoring local weather data.\\
\textbf{Preconditions:  }The System is running, the heat index
data is calculated (see \emph{Use Case S5--Calculate Heat Index}). The
extremum heat index data values are initialized: -999.9 for max, 999.9
for min.\\
\textbf{Postconditions:  }The extremum heat index data (minumum
and maximum) are  noted and coverted to three units (Celsius,
Fahrenheit, Kelvin).\\\\
\textbf{Flow of Events}\\
\begin{tabular}{|l|l|}\hline
\textbf{System} & \textbf{One Wire Temperature Sensor}\\\hline
& 1.  Return requested temperature\\ 
&      and humditiy data  measured\\\hline
2.     Calculates the heat index (See & \\
 \emph{Use Case S5--Calculate Heat Index}) & \\\hline
3.     Records the date/time of the & \\
current heat index measurement. & \\\hline
4.     Compare the current heat index & \\
against both the minimum and maximum. &\\\hline
5.     Record the current heat index as an &\\
extreme if it is less than or equal to the &\\
minimum or greater than or equal &\\
to the maximum. & \\\hline
6.    Record the date and time for & \\
\emph{event 5} & \\\hline
\end{tabular}\\\\
\textbf{Exception Conditions}\\
1a.  If either the One Wire temperature or the hygrometer sensor
stops working, then the System changes neither the minimum nor
maximum values.\\
1b.  If either the One Wire temperature  or the hygrometer sensor
returns an error, then the System changes neither the minumum nor
maximum values.\\
1c.  If the One Wire Network breaks, then the System changes neither
the minumum nor maximum values.\\
4a-5a.  If the data has changed from the previous messurement (must be
compared), the  heat index extremum are reset to the default
values and compared for the new date.
\end{document}
