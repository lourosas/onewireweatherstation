\documentclass[letterpaper]{article}
\usepackage{latexsym}
\usepackage{amsbsy}
\usepackage{graphicx}
\begin{document}
\noindent
\textbf{Use Case S2:  Detailed Description}\\\\
Use Case Name:  Monitor Humidity Data\\
Scenario:  Record Daily Humdity Extremes\\
Brief Description:  With the System running, the System records the
daily humidity extreme data--the minimum and maximum humidity--from
the 1-wire humidity sensor.\\
Actors:  System\\
Related Use Cases:\\
\textbf{Use Case S9:  }\emph{The System Shall Save Humidity
Extremes}\\
\textbf{Stakeholders:  }Local and National Weather Bureaus and
individuals monitoring local weather data.\\
\textbf{Preconditions:  }The System is running, the humdity data
is received from the hygrometer hardware. The extreme humidity
data values are initialized: -1.0 for max, 101.0 for min.\\
\textbf{Postconditions:  }The extreme humidity (minumum and
maximum) data
found and noted.\\\\
\textbf{Flow of Events}\\
\begin{tabular}{|l|l|}\hline
\textbf{System} & \textbf{One Wire Hygrometer Sensor}\\\hline
& 1.  Return requested humidity\\ 
&      data  measured\\\hline
2.    Records the date/time of the & \\
       current humidity  measurement. & \\\hline
3.     Compare the current humidity & \\
against both the minimum and & \\
maximum. &\\\hline
4.     Record the current temperature & \\
as an extreme if it is less than or & \\
equal to the minimum or greater & \\
than or equal to the maximum. & \\\hline
5.    Record the date and time for & \\
\emph{event 4} & \\\hline
\end{tabular}\\\\
\textbf{Exception Conditions}\\
1a.  If the One Wire Hygrometer sensor stops working, then the System
changes neither the minimum nor maximum values.\\
1b.  If the One Wire Hygrometer sensor returns an error, then the
System changes neither the minumum nore maximum values.\\
1c.  If the One Wire Network breaks, then the System changes neither
the minumum nor maximum values.\\
3a-5a.  If the date has changed from the previous measurement, the
humidity max and min are reset and compared for the new date.
\end{document}