\documentclass[letterpaper]{article}
\usepackage{latexsym}
\usepackage{amsbsy}
\usepackage{graphicx}
\begin{document}
\noindent
\textbf{Use Case S3:  Detailed Description}\\\\
Use Case Name:  Monitor Barometric Preasure\\
Scenario:  N/A\\
Brief Description:  With the System running, the System monitors the
barometric preasure by periodically requesting raw voltage data from
the 1-Wire Barometric Pressure Sensor.\\
\textbf{Note: }\emph{The 1-Wire Barometric Pressure Sensor is
essentially a Specially Modified \underline{DS2438 Smart Battery
Monitor} that can be used to measure Barometric Pressure}.\\
Actors:  System\\
Related Use Cases:\\
\textbf{Use Case A1:  }\emph{The Administrator shall set the
Measurement Rate for all acvite hardware sensors}\\
\textbf{Use Case S12:  }\emph{The System Shall Save the Barometric
Pressure data}\\
\textbf{Use Case S13:  }\emph{The System Shall Monitor and Save
Barometric Extreemes}\\
\textbf{Stakeholders:  } Local and National Weather Bureaus, other
systems and users monitoring local weather data.\\
\textbf{Preconditions:  } The System is running, the Measurment rate
is set, the 1-Wire Network is setup and working, the barometer
hardware is connected to the network and working properly.\\
\textbf{Postconditions: } The Raw Barometric Pressure data is
received from the Barometric Pressure Sensor Hardware and converted
to actual barometric pressure data.\\\\
\textbf{Flow of Events}\\
\begin{tabular}{|l|l|}\hline
\textbf{System} & \textbf{One Wire Barometric Sensor}\\\hline
1.  Periodically request raw barometric  & \\
pressure data from the One Wire Barometric & \\ Pressure
Sensor & 2.  Returns the raw Barometric Pressure \\ &  data
(the supply and input voltages)\\\hline
3.  Converts the raw Barometric Data & \\
(input and supply voltages) into & \\
barometric pressure & \\\hline
\end{tabular}
\end{document}