%%%%%%%%%%%%%%%%%%%%%%%%%%%%%%%%%%%%%%%%%%%%%%%%%%%%%%%%%%%%%%%%%%%%%
\documentclass{article}
\usepackage{latexsym}
\usepackage{amsbsy}
\usepackage{graphicx}
\begin{document}
\noindent
\textbf{Use Case S6:  Detailed Description}\\\\
Use Case Name:  Save Meteorological Data\\
Scenario:  Save Temperature Data to a Database\\
Brief Description:  With the System running, and the database server
working, the system sends the Temperature Data to the database for
archiving--including the current minimum and maximum.\\
Primary Actor:  System\\
Secondary Actor:  Database\\
Related Use Cases:  \textbf{Use Case S1:  }
\emph{The System Shall Monitor Temperature Data}\\
\textbf{Use Case S6 Scenario 1:  }
\emph{The System Shall Save Temperature Data}\\
\textbf{Stakeholders:  }Local and National Weather Buereaus and
individuals desiring to archive temperature data.\\Climatoligists
interested in archived temperature data.\\\\
\textbf{Preconditions:  }The System is running, the temperature is
monitored, the database server is running the System is logged
in/accessing the database.\\\\
\textbf{Postconditions:  }The data is archived to the database.\\\\
\textbf{Flow of Events}\\
\begin{tabular}{|l|l|}\hline
\textbf{System} & \textbf{Database}\\\hline
1.  Send Temperature Data to &\\
Database &\\\hline
& 2.  Parse the Date into Month, Year, Day\\
& fields\\\hline
& 3.  Place the Time data in the perspective\\
& Day field.\\\hline
& 4.  Parse the temperature data into Metric,\\
& English and Absolute\\\hline
& 5.  Place the Metric Temp Data in the\\
&  Metric Field, the English Temp Data in\\
&  the English Field, the Absolute Temp\\
&  Data in the Absolute Field\\\hline
6.  Send the Max/Min Temperature data &\\
to the Database with the Date. &\\\hline
& 7.  Parse the Date into Month, Year, Day\\
& fields\\\hline
& 8.  Plase the Time data in the perspective\\
& Day field.\\\hline
& 9.  Parse the max/min temperature into the\\
& appropriate units\\\hline
& 10.  Place the Max/Min temperature data into\\
& the appropriate Max/Min with the appropriate\\
& units.\\\hline
\end{tabular}
\\\textbf{Exception Conditions}\\
1a, 6a.  If the Database Server is not running, then the System
cannot save the data to the Database and alerts.\\\\
*a. At anytime, if the fields do not contain the current data for
saving, then the Database creates the data:  placing that data in
the appropriate Entity for the Database.\\\\
*b. At anytime, if the data cannot be saved in the Database, then the
Database alerts the System, the System alerts.\\\\
\textbf{Technology and Data Variations List}\\
*a.  The Database Could be one of several different types:
preferably an SQL type Database.
*b.  Querrying of the database via GUI interaction (if not already
implemented) to be developed within 6 months.
\end{document}
