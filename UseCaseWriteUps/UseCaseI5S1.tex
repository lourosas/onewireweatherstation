\documentclass[letterpaper]{article}
\usepackage{latexsym}
\usepackage{amsbsy}
\usepackage{graphicx}
\begin{document}
\noindent
\begin{center}
\textbf{Use Case I5S1:  Detailed Description}
\end{center}
Use Case Name:  Manage Calculated Weather Data\\
Scenario:  S1:  Manage Heat Index Data\\
Brief Description:  With the System Running, the User requests the
Heat Index Data from the System.  The System calculates the Heat
Index from the the temperature and humidity mission data, returning
it to the User.\\
Actors:  User\\
Related Use Cases: \textit{Use Case I4:  The User shall manage
mission data}\\\\
Stakeholders:  Users who want to manage the current Heat Index Data.
\\\\
Preconditions:  The System is running, the iButton is connected to
the receptor or another reading device.  The iButton receptor or
other reading device is connected to the computer.\\\\
Postconditions:  The Heat Index Data is calculated and returned to
the User. The User can chose how to archive the data or evaluate the
data imediately.\\\\
\textbf{Flow of Events}\\\\
\begin{tabular}{|l|l|}\hline
\textbf{User} & \textbf{System}\\\hline
1.  Requests Heat Index Data & \\\hline
2.  Requests the Units for displaying &\\
the Heat Index Data (Celsius, Fahrenheit &\\
Kelvin) & \\\hline
 & 3.  Retrieves the temperature data\\
 & (See \textbf{Use Case I4}), Retrieves\\
 & the humidity data (See \textbf{Use Case I4})\\\hline
 & 4.  Applies the appropriate \\
 & \underline{Heat Index Calculation}\\\hline
 & 5.  Converts the data to the appropriate\\
 & units as requested by the User.\\\hline
\end{tabular}\\\\
\textbf{Exception Conditions}\\
3a.  If either the temperature or humidity data or both could not
be retrieved for the heat index calculation, then the system
indicates the error by setting the Heat Index to a defualt value
(NaN) as well as a possible reason for the error (One Wire Network
or One Wire Device issues).\\\\
3b.  If the mission record time for either the temperature or
humidity data are not the same, then the system indicates the
error by setting the Heat Index to a default value (NaN).\\\\
4a.  If the temperature is too low for an accurate Heat Index
Calculation (temperature $< 70^{\circ}F$), then the system indicates
the issue by setting the Heat Index to a default value (NaN).\\\\
4b.  If the relative humidity is too low for an accurate Heat Index
calculation (humidity $< 0\%$), then the system indicates the issue
by setting the Heat Index to a default value (NaN).
\end{document}
